%-------------------------------------------------------------------------------
%	SECTION TITLE
%-------------------------------------------------------------------------------
\cvsection{Experience}


%-------------------------------------------------------------------------------
%	CONTENT
%-------------------------------------------------------------------------------
\begin{cventries}

%---------------------------------------------------------
  \cventry
    {http://www.amazon.co.jp/}
    {Senior Applied Scientist} % Job title
    {Promoted from Applied Scientist II in Jun. 2024}
    {Amazon} % Organization
    {Tokyo, Japan} % Location
    {Apr. 2020 - Present} % Date(s)
    {
      \begin{cvitems} % Description(s) of tasks/responsibilities
        \item {Spearheaded 5+ end-to-end ML projects that established a strategic charter in Japan, launching innovative programs and personally driving their global expansion through managing \underline{50+ stakeholders} (including directors and VPs) across JP, US, IN, and CN.}
        \item {Engineered production-grade pipelines using AWS leveraging orchestration, serverless processing, and real-time monitoring with solutions in Python, SQL, and Scala—to deliver scalable, end-to-end ML systems.}
        \item {Authored an influential ML paper presented at our internal conference that introduced a novel contrastive learning framework to fine-tune item embeddings within the context of \textit{replaceability}. Implementation resulted in a \underline{26.2\%} improvement in coverage of an internal dataset.}
        \item {Trained \& validated multiple causal inference models to deliver \underline{6M+} action recommendations to vendors across Amazon globally daily, with a combined potential estimate of \underline{\$920M+ USD} uplift if executed on by vendors.}
        \item {Realised an annual uplift of \underline{\$21M+ USD} through an ML-based improvement in JP address resolution, reducing failed deliveries by \underline{81.2\%}.}
      \end{cvitems}
    }


%---------------------------------------------------------
  \cventry
    {http://www.ascent.ai/en/}
    {Team Lead Research Engineer} % Job title
    {Promoted from Research Engineer in Apr. 2019}
    {Ascent Robotics} % Organization
    {Tokyo, Japan} % Location
    {May 2017 - Feb. 2020} % Date(s)
    {
      \begin{cvitems} % Description(s) of tasks/responsibilities
       % \item {Lead the \emph{Adversary} team which focused on validating the effectiveness of Ascent's autonomous vehicles over all possible scenarios through the use of simulation and advanced adversarial search techniques.}
        \item {Led the \textit{Adversarial AI} team of \underline{8} other scientists, focusing on exposing and proposing solutions to incapabilities in the vehicle AI through simulation prior to testing in the real world.}
      % \item {Lead multiple end-to-end machine learning-based projects specifically with the goal of training our vehicles to adopt a human-like behaviour. The results of these projects have been used in publications at leading conferences worldwide.}
       % \item {Created a full end-to-end stack which clusters driving behaviours of vehicles on Japanese roads from raw 4K HD top-down drone video data. This data has been converted to an API and used in various other projects within the Adversary and vehicle AI team.}
        \item {Launched a foundational dataset of driver-behaviour clusters to aid adversarial simulation. Clusters were learned from hundreds of hours of 4K aerial drone footage over various highways in Japan. Employed a combination of ML techniques in computer vision \& reinforcement learning.}
      \end{cvitems}
    }

%---------------------------------------------------------
  \cventry
    {https://www.informetis.com/en/}
    {Research Engineer} % Job title
    {UCL Master's Thesis}
    {Informetis} % Organization
    {Tokyo, Japan} % Location
    {June. 2018 - Sept. 2018} % Date(s)
    {
      \begin{cvitems} % Description(s) of tasks/responsibilities
        \item {Developed an optimal control policy for \emph{peak-demand} energy management as part of my Master's thesis at at \emph{UCL}, employing reinforcement learning (PPO) to strategically shift consumer demand and optimise battery storage.}
        \item {Achieved perfect peak-shifting, reducing energy consumption and utility costs by over \underline{20\%}.}
        %\item {The code and full thesis is available on \href{https://github.com/KeirSimmons/RL-Optimal-Peak-Shift}{\emph{GitHub}}.} 
      \end{cvitems}
    }

%---------------------------------------------------------
  \cventry
    {http://www.airc.aist.go.jp/en/}
    {Machine Learning Researcher} % Job title
    {Courtesy of the Daiwa Scholarship}
    {AI Research Center - National Institute of Advanced Industrial Science \& Technology} % Organization
    {Tokyo, Japan} % Location
    {Oct. 2016 - Mar. 2017} % Date(s)
    {
      \begin{cvitems} % Description(s) of tasks/responsibilities
        \item {Earned a competitive research placement as the only non-doctorate researcher, implementing core ML algorithms—including reproducing \& fine-tuning DeepMind's Atari results.}
        \item {Led and delivered a reinforcement learning reading group for \underline{50+} researchers to study the \textit{Reinforcement Learning} book by Sutton \& Barto.}
      \end{cvitems}
    }

%---------------------------------------------------------
  \cventry
    {http://www.dajf.org.uk/scholarships/daiwa-scholarship/scholars/daiwa-scholars-2015}
    {Daiwa Scholar} % Job title
    {}
    {The Daiwa Anglo-Japanese Foundation} % Organization
    {Tokyo, Japan} % Location
    {Aug. 2015 - Mar. 2017} % Date(s)
    {
      \begin{cvitems} % Description(s) of tasks/responsibilities
        \item {Awarded the prestigious Daiwa Scholarship, a unique and highly competitive programme valued at approximately \underline{11M JPY}. I was selected as \underline{one of six} scholars from over \underline{1,000} candidates to represent the United Kingdom in machine learning in Japan.}
        \item {Completed an intensive language course, achieving business level fluency in Japanese (JLPT N2).}
      \end{cvitems}
    }

%---------------------------------------------------------
  \cventry
    {https://www.eng.nus.edu.sg/undergraduatestudies/academics/enhancement-programmes/undergraduate-research-opportunities-programme-urop/}
    {Undergraduate Researcher} % Job title
    {}
    {National University of Singapore} % Organization
    {Kent Ridge, Singapore} % Location
    {Jan. 2014 - Jun. 2014} % Date(s)
    {
      \begin{cvitems} % Description(s) of tasks/responsibilities
        \item {Developed a custom sound pressure level meter application for Android using Java to monitor aircraft engine noise, incorporating a real-time graphical frequency analyser that meets aviation industry standards.}
      \end{cvitems}
    }

%---------------------------------------------------------
  \cventry
    {} % link to github after porting everything...
    {Web Developer} % Job title
    {}
    {Volunteering} % Organization
    {London, UK} % Location
    {Jan. 2007 - Dec. 2014} % Date(s)
    {
      \begin{cvitems} % Description(s) of tasks/responsibilities
        \item {Created over \underline{100} advanced open-source modifications in JavaScript and PHP for use with the popular message-board services \textit{Invision Power Board} and \textit{ZetaBoards}, including one that was used by over \underline{1M people}. Some have now been ported to \href{https://github.com/KeirSimmons/ZetaBoards}{\emph{GitHub}} for posterity.}
      \end{cvitems}
    }

%---------------------------------------------------------
%  \cventry
%    {http://aerosoc.co.uk/}
%    {Aeronautical Society Webmaster} % Job title
%    {Imperial College London} % Organization
%    {London, UK} % Location
%    {Sept. 2012 - Jul. 2013} % Date(s)
%    {
%      \begin{cvitems} % Description(s) of tasks/responsibilities
%        \item {Overhauled the society's website through the use of PHP and JavaScript, writing efficient, concise and reusable code which was well commented and structured for ease of development. Website design changes annually.}
%      \end{cvitems}
%    }

%---------------------------------------------------------

%---------------------------------------------------------
\end{cventries}